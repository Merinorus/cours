\documentclass[11pt]{article}
\usepackage[utf8]{inputenc}
\usepackage[T1]{fontenc}
\usepackage[francais]{babel}
\usepackage[top=2.5cm, bottom=2.5cm, left=2.5cm, right=2.5cm]{geometry}
\usepackage{hyperref}   % Sommaire PDF
\usepackage{amsmath}    % Symboles mathématiques
\usepackage{amssymb}    % Flèches barrées
\usepackage{graphicx}   % Insertion d'images
\usepackage{multirow}   % Fusion de cellules de tableaux
\usepackage{float}      % Placement des figures
\usepackage{xcolor}     % Couleurs
\usepackage{eurosym}    % Symbole €
\usepackage{listings}	% Listings
\usepackage{alltt}		% Verbatim

% En-tête, pied-de-page
\usepackage{fancyhdr}
\pagestyle{fancy}
\renewcommand{\headrulewidth}{0pt}
\lhead{}
\chead{}
\rhead{}
\lfoot{}
\cfoot{\thepage}
\rfoot{}

\lstset{
    frame=single,
    breaklines
}

\begin{document}
    \title{Algèbre de Boole}
    \author{Michel BERNE}
    \date{\today}

% Page de garde + page blanche
\maketitle
\setcounter{page}{0} \thispagestyle{empty} % Ne pas numéroter la page de garde + pas d'en-tête/pied-de-page
\newpage
\setcounter{page}{2} % Commencer la numérotation à 2
~\\
\newpage

\renewcommand{\contentsname}{Sommaire}
\tableofcontents

% Début du document
\newpage
\part{Relation d'ordre}
    \section{Définition}
        Soit $E$ un ensemble et $R$ une relation linéaire entre éléments de $E$. $R$ est un ordre sur $E \Leftrightarrow R$ est réflexive, antisymétrique et transitive.
        \begin{itemize}
            \item réflexive : $\forall a\in E,aRa$
            \item antisymétrique : $\forall a,b\in E,aRb\land bRa\Rightarrow a=b$
            \item transitive : $\forall a,b,c\in E,aRc$
        \end{itemize}

        \paragraph{Remarque} Si pour tous $a,b\in E$ on a $(aRb)\lor(bRa)$, l'ordre est total. Sinon, l'ordre est partiel (certains éléments ne sont pas « comparables » par $R$ car on a ni $(aRb)$ ni $(bRa)$.


        \paragraph{Exemples}
            \begin{enumerate}
                \item Soit $E\neq\emptyset$ : l'inclusion définit un ordre (partiel) sur $P(E)$. $A\subset B\Leftrightarrow\forall x,x\in A\Leftrightarrow x\in B$

                On n'a ni $A\subset B$, ni $B\subset A$, l'ordre est partiel. D'autre part, on a bien :
                \begin{itemize}
                    \item $A\subset A$ pour toute $A\in P(E), C$ est réflexive
                    \item $(A\subset B$ et $B\subset A)\Rightarrow A=B$ : c'est la définition même de $A=B$
                    \item $(A\subset B$ et $B\subset C)\Rightarrow A\subset C$
                \end{itemize}

                $C$ est asymétrique, $\subset$ est donc un ordre de $P(E)$.

                \item La relation « $a$ divise $b$ » définit un ordre partiel sur $\mathbb{N}^*$ (notation : $a|b$). En effet, pour tout $a\neq 0, a|a$ donc $|$ est réflexive. $(a|b)$ et $(b|a)\Rightarrow a=b$ donc $|$ est asymétrique. $(a|b)$ et $(b|a)\Rightarrow a|c$ donc $|$ est transitive. Par exemple, $5|15$ mais $\lnot(4|15)$ : l'ordre est donc partiel.
                \item Représenter la relation $a|b$ sur l'ensemble des diviseurs de 30, puis 60.
                    \begin{itemize}
                        \item Les diviseurs de 30 : ${1,2,3,5,6,10,15,30}$
                        \item Les diviseurs de 60 : ${1,2,3,4,5,6,10,12,15,20,30,60}$
                    \end{itemize}

                Sur $\mathbb{N}$, la relation $\leq$ usuelle définit un ordre total.
            \end{enumerate}

    \section{Éléments remarquables liés à une relation d'ordre}
\end{document}
