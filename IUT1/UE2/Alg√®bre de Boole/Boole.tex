\documentclass[11pt]{article}
\usepackage[utf8]{inputenc}
\usepackage[T1]{fontenc}
\usepackage[francais]{babel}
\usepackage[top=2.5cm, bottom=2.5cm, left=2.5cm, right=2.5cm]{geometry}
\usepackage{hyperref}   % Sommaire PDF
\usepackage{amsmath}    % Symboles mathématiques
\usepackage{amssymb}    % Flèches barrées
\usepackage{graphicx}   % Insertion d'images
\usepackage{multirow}   % Fusion de cellules de tableaux
\usepackage{float}      % Placement des figures
\usepackage{xcolor}     % Couleurs
\usepackage{eurosym}    % Symbole €
\usepackage{listings}	% Listings
\usepackage{alltt}		% Verbatim

% En-tête, pied-de-page
\usepackage{fancyhdr}
\pagestyle{fancy}
\renewcommand{\headrulewidth}{0pt}
\lhead{}
\chead{}
\rhead{}
\lfoot{}
\cfoot{\thepage}
\rfoot{}

\lstset{
    frame=single,
    breaklines
}

\begin{document}
    \title{Algèbre de Boole}
    \author{Michel BERNE}
    \date{\today}

% Page de garde + page blanche
\maketitle
\setcounter{page}{0} \thispagestyle{empty} % Ne pas numéroter la page de garde + pas d'en-tête/pied-de-page
\newpage
\setcounter{page}{2} % Commencer la numérotation à 2
~\\
\newpage

\renewcommand{\contentsname}{Sommaire}
\tableofcontents

% Début du document
\newpage
\part{Relation d'ordre}
    \paragraph{Définition}
        Soit $E$ un ensemble et $R$ une relation linéaire entre éléments de $E$. $R$ est un ordre sur $E \Leftrightarrow R$ est réflexive, antisymétrique et transitive.
        \begin{itemize}
            \item réflexive : $\forall a\in E,aRa$
            \item antisymétrique : $\forall a,b\in E,aRb\land bRa\Rightarrow a=b$
            \item transitive : $\forall a,b,c\in E,aRc$
        \end{itemize}

\end{document}
