\documentclass[11pt]{article}
\usepackage[utf8]{inputenc}
\usepackage[T1]{fontenc}
\usepackage[francais]{babel}
\usepackage[top=2.5cm, bottom=2.5cm, left=2.5cm, right=2.5cm]{geometry}
\usepackage{hyperref}   % Sommaire PDF
\usepackage{amsmath}    % Symboles mathématiques
\usepackage{amssymb}    % Flèches barrées
\usepackage{graphicx}   % Insertion d'images
\usepackage{multirow}   % Fusion de cellules de tableaux
\usepackage{float}      % Placement des figures
\usepackage{xcolor}     % Couleurs
\usepackage{eurosym}    % Symbole €
\usepackage{listings}	% Listings
\usepackage{alltt}		% Verbatim

% En-tête, pied-de-page
\usepackage{fancyhdr}
\pagestyle{fancy}
\renewcommand{\headrulewidth}{0pt}
\lhead{}
\chead{}
\rhead{}
\lfoot{}
\cfoot{\thepage}
\rfoot{}

\lstset{
    frame=single,
    breaklines
}

\begin{document}
    \title{Algèbre de Boole}
    \author{Michel BERNE}
    \date{\today}

% Page de garde + page blanche
\maketitle
\setcounter{page}{0} \thispagestyle{empty} % Ne pas numéroter la page de garde + pas d'en-tête/pied-de-page
\newpage
\setcounter{page}{2} % Commencer la numérotation à 2
~\\
\newpage

\renewcommand{\contentsname}{Sommaire}
\tableofcontents

% Début du document
\newpage
\part{Relation d'ordre}
    \section{Définition}
        Soit $E$ un ensemble et $R$ une relation linéaire entre éléments de $E$. $R$ est un ordre sur $E \Leftrightarrow R$ est réflexive, antisymétrique et transitive.
        \begin{itemize}
            \item réflexive : $\forall a\in E,aRa$
            \item antisymétrique : $\forall a,b\in E,aRb\land bRa\Rightarrow a=b$
            \item transitive : $\forall a,b,c\in E,aRc$
        \end{itemize}

        \paragraph{Remarque} Si pour tous $a,b\in E$ on a $(aRb)\lor(bRa)$, l'ordre est total. Sinon, l'ordre est partiel (certains éléments ne sont pas « comparables » par $R$ car on a ni $(aRb)$ ni $(bRa)$.


        \paragraph{Exemples}
            \begin{enumerate}
                \item Soit $E\neq\emptyset$ : l'inclusion définit un ordre (partiel) sur $P(E)$. $A\subset B\Leftrightarrow\forall x,x\in A\Leftrightarrow x\in B$

                On n'a ni $A\subset B$, ni $B\subset A$, l'ordre est partiel. D'autre part, on a bien :
                \begin{itemize}
                    \item $A\subset A$ pour toute $A\in P(E), C$ est réflexive
                    \item $(A\subset B$ et $B\subset A)\Rightarrow A=B$ : c'est la définition même de $A=B$
                    \item $(A\subset B$ et $B\subset C)\Rightarrow A\subset C$
                \end{itemize}

                $C$ est asymétrique, $\subset$ est donc un ordre de $P(E)$.

                \item La relation « $a$ divise $b$ » définit un ordre partiel sur $\mathbb{N}^*$ (notation : $a|b$). En effet, pour tout $a\neq 0, a|a$ donc $|$ est réflexive. $(a|b)$ et $(b|a)\Rightarrow a=b$ donc $|$ est asymétrique. $(a|b)$ et $(b|a)\Rightarrow a|c$ donc $|$ est transitive. Par exemple, $5|15$ mais $\lnot(4|15)$ : l'ordre est donc partiel.
                \item Représenter la relation $a|b$ sur l'ensemble des diviseurs de 30, puis 60.
                    \begin{itemize}
                        \item Les diviseurs de 30 : ${1,2,3,5,6,10,15,30}$
                        \item Les diviseurs de 60 : ${1,2,3,4,5,6,10,12,15,20,30,60}$
                    \end{itemize}

                Sur $\mathbb{N}$, la relation $\leq$ usuelle définit un ordre total.
            \end{enumerate}

    \section{Éléments remarquables liés à une relation d'ordre}
        \subsection{Minorants et majorants}
        Soit $E$ un ensemble ordonné par $R$ et $A\subset E$ :
            \begin{itemize}
                \item $m\in E$ est un minorant de $A$ pour $R\Leftrightarrow\forall a\in A,mRa$
                \item $M\in E$ est un majorant de $A$ pour $R\Leftrightarrow\forall a\in A, aRM$
            \end{itemize}

        \subsection{Éléments minimaux et maximaux}
            $a\in A$ est minimal dans $A$ pour $R\Leftrightarrow\forall x,(x\in A$ et $xRa)\Rightarrow x=a$, et maximal dans $A$ pour $R\Leftrightarrow\forall x,(x\in A$ et $aRx)\Rightarrow x=a$.

            Un élément minimal dans $A$ n'a pas de minorant strict dans $A$. Un élément maximal dans $A$ n'a pas de majorant strict dans $A$.

        \subsection{Plus petit élément, plus grand élément}
            $A\subset E$ admet un plus petit élément (noté $min(A)$)$\Leftrightarrow\exists a\in A,\forall x\in A, aRx$. $A\subset E$ admet un plus grand élément (noté $max(A)$)$\Leftrightarrow\exists a\in A,\forall x\in A, xRa$.

            \paragraph{Remarques}
                \begin{itemize}
                    \item $min(A)$ existe $\Leftrightarrow$ il existe un minorant de $A$ qui appartient à $A$ (c'est lui, $min(A)$)
                    \item $max(A)$ existe $\Leftrightarrow$ il existe un majorant de $A$ qui appartient à $A$ (c'est lui, $max(A)$)
                    \item Si $min(A)$ existe, il est unique; en effet, si $a=min(A)$ et $b=min(A)$, on doit voir $aRb$ et $bRa$, d'où $a=b$ par antisymétrie
                    \item Si $max(A)$ existe, il est unique; en effet, si $a=max(A)$ et $b=max(A)$, on doit voir $bRa$ et $aRb$, d'où $a=b$ par antisymétrie
                \end{itemize}

        \subsection{Borne inférieure et borne supérieure}
            $A\subset E$ admet une borne inférieure (notée $inf(A)$)$\Leftrightarrow A$ admet des minorants dans $E$ et l'ensemble des minorants admet un plus grand élément (c'est $inf(A)$).

            $A\subset E$ admet une borne supérieure (notée $sup(A)$)$\Leftrightarrow A$ admet des majorants dans $E$ et l'ensemble des majorants admet un plus petit élément (c'est $asup(A)$).

            \paragraph{Exemple} Dans les diviseurs de 30, ${1,3,5}$ est majoré par 15 et 30; ${5,10,15}$ est minoré par 5 et 1.

\newpage
\part{Treillis}
    Dorénavant, on note par défaut $\leq$ pour une relation d'ordre, et $a<b$ pour ($a\leq b$ et $a\neq b$).

    \section{Définition}
        $E$ muni d'un ordre $\leq$ est un treillis $\Leftrightarrow$ quels que soit $a,b$ dans $E$, il existe $inf(a,b)$ et $sup(a,b)$.

        \paragraph{Remarque} Si $\leq$ est un ordre total, on a $a\leq b$ et $b\leq a$ :
            \begin{itemize}
                \item dans le 1\up{er} cas : $a=inf(a,b)$ et $b=sup(a,b)$
                \item dans le 2\up{ème} cas : $a=sup(a,b)$ et $b=inf(a,b)$
            \end{itemize}

        Les treillis intéressants sont plutôt ceux pour lequel l'ordre $\leq$ est partiel. Par exemple, $E={1,2,3,5,6,10,10,15,30}$ est un treillis pour la divisibilité : $inf(6,10)=2sup(6,10)=30$ et $inf(3,5)=1sup(3,5)=15$.

        \paragraph{Exemples}
            \begin{enumerate}
                \item $E={a,b,c,d}\leq$ défini par $\left|\begin{matrix}a\leq c\\b\leq c\\a\leq d\\b\leq d\end{matrix}\right.$ plus la réflexivité et la « fermeture transitive » $\leq$ est bien un ordre (par construction), mais $(E,\leq)$ n'est pas un treillis : ${c,d}$ n'est pas majoré donc $sup{c,d}$ n'existe pas. $a$ et $b$ minorent ${c,d}$ mais ne sont pas comparables, donc $inf(c,d)$ n'existe pas.
                \item $\leq$ est définie par $x\leq y\Leftrightarrow$ il existe une suite de flèches adjacentes conduisant de $x$ à $y$ (plus la réflexivité). On a bien une structure de treillis et l'ordre est partiel (on n'a ni $b\leq c$ ni $c\leq b$).
                \item $E\neq\emptyset$ : $(P(E),\subset)$ est un treillis. Pour $a\subset E,B\subset E$ on a $inf(A,B)=A\cap B$ et $sup(A,B)=A\cup B$.
            \end{enumerate}

    \section{Opérations sur un treillis}
        Soit $(E,\leq)$ un treillis. On pose :
        $$
            a+b:=sup(a,b)
        $$$$
            a\times b:=inf(a,b)
        $$

    \section{Propriétés}
        \begin{enumerate}
            \item $+$ et $\times$ sont associatives et commutatives
            \item Si $E$ est fini alors $sup(E)=max(E)$ et $inf(E)=min(E)$ existent; si $E={R_1,R_2,\hdots,R_n}$, on pose :
            $$
                1_E=a_1+a_2+\hdots+a_n=max(E)
            $$$$
                0_E=a_1\times a_2\times\hdots\times a_n=min(E)
            $$
            \item Si $1_E=sup(E)$ et $0_E=inf(E)$ existent, alors :
                \begin{itemize}
                    \item $1_E$ est absorbant par $+$ : $a+1_E=1_E$ pour tout $a\in E$
                    \item $A_E$ est élément neutre par $\times$ : $a\times 1_E=a$ pour tout $a\in E$
                    \item $9_E$ est absorbant par $\times$ : $a\times 0_E=0_E$ pour tout $a\in E$
                    \item $0_E$ est élément neutre par $+$ : $a+0_E=a$ pour tout $a\in E$
                \end{itemize}
        \end{enumerate}

    \subsection{Treillis distributifs}
        Un treillis $(E,\leq)$ est dit distributif si :
            \begin{itemize}
                \item $(a+b)\times c=a\times c+b\times c$ (distributivité de $\times$ par rapport à $+$)
                \item $a+b\times c=(a+b)\times(a+c)$ (distributivité de $+$ par rapport $\times$)
            \end{itemize}
        
            \paragraph{Exemples}
                \begin{enumerate}
                    \item $E\neq\emptyset,(P(E),\subset)$ : ici, $+$ est la réunion et $\times$ est l'intersection, et on a :
                        \begin{itemize}
                            \item $((A\cup B)\cap C)=(A\cap C)\cup(B\cap C)$
                            \item $A\cup(B\cap C)=(A\cup B)\cap(A\cup C)$
                        \end{itemize}
                    \item $E={a,b,c,d,e}\leq$ induite par $\left|\begin{matrix}a\leq b\\
                    a\leq c\\
                    a\leq e\\
                    b\leq d\\
                    c\leq d\\
                    e\leq d\\
                    \end{matrix}\right.$

                    $(E,\leq)$ est bien un treillis, mais il n'est pas distributif :
                        $$
                        \left(
                        \begin{array}{rcrcrcrcr}
                            &&&&(e\times(b+c)) &=& e\times d &=&e\\
                            &&&&&& e\times b &=& a\\
                            e\times c &=& a &\Rightarrow& e\times b+e \times c &=&a+a &=& a
                        \end{array}
                        \right)\Rightarrow e\times b(b+c)\neq e\times b+e\times c
                        $$
                \end{enumerate}

    \subsection{Treillis complémentés}
        Soit $(E,\leq)$ un treillis avec $1_E=sup(E)$ et $0_E=inf(E)$; $(E,\leq)$ est dit complémenté si, pour chaque $a\in E$, il existe $x\in E$ vérifiant $\left|\begin{matrix}a+x=1_E\\a\times x=0_E\end{matrix}\right.$

        \paragraph{Remarque} Si $(E,\leq)$ est distributif, le système ci-dessus admet au plus une solution : on l'appelle le complément de $a$ et on le note $\bar{a}$. On aura alors évidemment $\bar{\bar{a}}=a$.

    \subsection{Algèbre de Boole}
        \subsubsection{Définition}
            Une algèbre de Boole est un treillis distibutif complémenté. L'exemple prototype $(P(E),\subset)$ avec $E\neq\emptyset$. Un exemple fondamental est $B={0,1}$.
\end{document}
