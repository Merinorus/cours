\documentclass[10pt]{article}
\usepackage[utf8]{inputenc}
\usepackage[T1]{fontenc}
\usepackage[francais]{babel}
\usepackage[top=2.5cm, bottom=2.5cm, left=2.5cm, right=2.5cm]{geometry}
\usepackage{hyperref}   % Sommaire PDF
\usepackage{amsmath}    % Symboles mathématiques
\usepackage{amssymb}    % Flèches barrées
\usepackage{graphicx}   % Insertion d'images
\usepackage{multirow}   % Fusion de cellules de tableaux
\usepackage{float}      % Placement des figures
\usepackage{xcolor}     % Couleurs
\usepackage{eurosym}    % Symbole €
\usepackage{listings}	% Listings
\usepackage{alltt}		% Verbatim

% En-tête, pied-de-page
\usepackage{fancyhdr}
\pagestyle{fancy}
\renewcommand{\headrulewidth}{0pt}
\lhead{Memento des ordres SQL*Plus}
\chead{}
\rhead{}
\lfoot{}
\cfoot{\thepage}
\rfoot{}

\begin{document}

\title{Memento des ordres SQL*Plus}
\date{}


% Sommaire
\newpage
\renewcommand{\contentsname}{Sommaire}
\maketitle

% Début du document	
\part{SQL comme langage de définition de données (LDD)}
	Types syntaxiques des attributs : \texttt{VARCHAR2(n) CHAR[(n)] NUMBER[(n[,m)]] DATE LONG}
	
	\section{Création de relation}
		\begin{alltt}
			\begin{tabbing}
				\emph{CREATE TABLE} \= <nom_table>\\
				\> (nom_colonne1> <type1> [\emph{DEFAULT} <expression1>]\\
				\>[, nom_colonne2> <type2> [\emph{DEFAULT} <expression2>]]\\
				\>[, <contrainte1>[, <contrainte2>]])
			\end{tabbing}
			où :
			\begin{tabbing}
				<contraintei> = \emph{CONSTRAINT} <nom_contrainte> <spec_contrainte> [<etat>]\\
				<spec_contrainte> = \= \emph{PRIMARY KEY}(<attribut1[, <attribut2>, ...])\\
									\> | \= \emph{FOREIGN KEY}(<attribut1>[, <attribut2>,...]) \\
									\> \> \emph{REFERENCES} <nom_relation_associée>(<attribut1>[, <attribut2>,...]) \\
									\> \> [\emph{ON DELETE <CASCADE|SET NULL>}]\\
									\> | \emph{CHECK}(<nom_attribut | expression> <condition>)\\
				<etat>= \emph{ENABLE} | \emph{DISABLE}
			\end{tabbing}
			\emph{CREATE TABLE} <nom_relation> [(liste_attributs>, <liste_contraintes>)]
			\emph{AS} <requete>
		\end{alltt}
		
	\section{Ajout d'attributs et de contraintes dans une relation}
		\begin{alltt}
			\begin{tabbing}
				\emph{ALTER TABLE} <nom_table>\\
				\emph{ADD} (\=[nom_colonne1> <type1>] [\emph{DEFAULT} <expression1>] [\emph{NOT NULL}] [\emph{UNIQUE}]\\
							\>[,<nom_colonne2> <type2> [\emph{DEFAULT} <expression2>] [\emph{NOT NULL}] [\emph{UNIQUE}]...]\\
							\>[,<contrainte1> ...])
			\end{tabbing}
			où
			\begin{tabbing}
				<contraintei> = \emph{CONSTRAINT} <nom_contrainte> <spec_contrainte> [<etat>]\\
				<spec_contrainte> = \= \emph{PRIMARY KEY}(<attribut1[, <attribut2>, ...])\\
									\> | \= \emph{FOREIGN KEY}(<attribut1>[, <attribut2>,...]) \\
									\> \> \emph{REFERENCES} <nom_relation_associée>(<attribut1>[, <attribut2>,...]) \\
									\> \> [\emph{ON DELETE <CASCADE|SET NULL>}]\\
									\> | \emph{CHECK}(<nom_attribut | expression> <condition>)\\
				<etat>= \= \emph{ENABLE} | \emph{DISABLE} | \emph{VALIDATE} | \emph{NOVALIDATE} | \emph{ENABLE VALIDATE} | \emph{ENABLE NOVALIDATE}\\
						\> | \emph{DISABLE VALIDATE} | \emph{DISABLE NOVALIDATE}
			\end{tabbing}
		\end{alltt}

	\section{Modification de la définition d'un attribut}
		\begin{alltt}
			\begin{tabbing}
				\emph{ALTER TABLE} <nom_table>\\
				\emph{MODIFY} \= [(]<nom_colonne1> [<nouveau_type1>] [\emph{DEFAULT} <expression1>] [\emph{NOT NULL}]\\
							  \> [,<nom_colonne2> [<nouveau_type2>] [\emph{DEFAULT} <expression2>] [\emph{NOT NULL}] ... [)]\\
			\end{tabbing}
		\end{alltt}
		
	\section{Modification de l'état d'une contrainte}
		\begin{alltt}
			\emph{ALTER TABLE} <nom_table>
			\emph{MODIFY CONSTRAINT} <nom_contrainte> <etat_contrainte>
		\end{alltt}
		
	\section{Suppression de contrainte dans une relation}
		\begin{alltt}
			\emph{ALTER TABLE} <nom_table> \emph{DROP CONSTRAINT} <nom_contrainte> [\emph{CASCADE}]
			
			\emph{ALTER TABLE} <nom_table> \emph{DROP UNIQUE}(<nom_attribut>) [\emph{CASCADE}]
			
			\emph{ALTER TABLE} <nom_table> \emph{DROP PRIMARY KEY} [\emph{CASCADE}]
		\end{alltt}
		
	\section{Suppression d'attribut dans une relation}
		\begin{alltt}
			\emph{ALTER TABLE} <nom_table> \emph{SET UNUSED COLUMN} <nom_attribut>
			
			\emph{ALTER TABLE} <nom_table> \emph{SET UNUSED}(<nom_attribut1>[, <nom_attribut2> ...])
			
			\emph{ALTER TABLE} <nom_table> \emph{DROP COLUMN} <nom_attribut> [\emph{CASCADE CONSTRAINTS}]
			
			\emph{ALTER TABLE} <nom_table> \emph{DDROP}(<nom_attribut1>[, <nom_attribut2>, ...]) [\emph{CASCADE CONSTRAINTS}]
			
			\emph{ALTER TABLE} <nom_table> \emph{DROP UNUSED COLUMNS}
		\end{alltt}
		
	\section{Suppresion de relation}
		\begin{alltt}
			\emph{DROP TABLE} <nom_table> [\emph{CASCADE CONSTRAINTS}]
		\end{alltt}
		
	\section{Création/suppression de synonyme et changement du nom d'une relation}
		\begin{alltt}
			\emph{CREATE} [\emph{PUBLIC}] \emph{SYNONYM} <nom_synonyme> \emph{FOR} <nom_objet>
			
			\emph{DROP SYNONYM} <nom_synonyme>
			
			\emph{RENAME} <ancien_nom> \emph{TO} <nouveau_nom>
		\end{alltt}
		
	\section{Gestion de séquences}
		\begin{alltt}
			\begin{tabbing}
				\emph{CREATE SEQUENCE} <nom_sequence> \= [\emph{START WITH} <valeur_initiale>]\\
													  \> [\emph{INCREMENT BY} <valeur_increment>]\\
													  \> [\emph{MAXVALUE} <valeur_maximale> | \emph{NOMAXVALUE}]\\
													  \> [\emph{MINVALUE} <valeur_minimale> | \emph{NOMINVALUE}]\\
													  \> [\emph{CYCLE} | \emph{NOCYCLE}]
			\end{tabbing}
			
			\emph{DROP SEQUENCE} <nom_sequence>
			
			\begin{tabbing}
				\emph{ALTER SEQUENCE} <nom_sequence> \= [\emph{INCREMENT BY} <valeur_increment]\\
													 \> [\emph{MAXVALUE} <valeur_maximale> | \emph{NOMAXVALUE}]\\
													 \> [\emph{MINVALUE} <valeur_minimale> | \emph{NOMINVALUE}]\\
													 \> [\emph{CYCLE} | \emph{NOCYCLE}]
			\end{tabbing}
		\end{alltt}
		
	\section{Index sur les relations}
		\begin{alltt}
			\emph{CREATE} [\emph{UNIQUE} | \emph{BITMAP}] \emph{INDEX} <nom_index>
			\emph{ON} <nom_table> ([<nom_colonne1>[, <nom_colonne2>, ...])
			
			\emph{ALTER TABLE} <nom_index> \emph{RENAME TO} <nouveau_nom>
			
			\emph{DROP INDEX} <nom_index>
		\end{alltt}
		
	\section{Principales tables systèmes Oracle}
		\begin{alltt}
			\begin{tabbing}
				ALL_CONS_COLUMNS \= (OWNER, CONSTRAINT_NAME, TABLE_NAME, COLUMN_NAME,...)\\
				ALL_CONSTRAINTS \> (OWNER, CONSTRAINT_NAME, CONSTRAINT_TYPE, TABLE_NAME, SEARCH_CONDITION,...)\\
				ALL_INDEXES \> (OWNER, INDEX_NAME, INDEX_TYPE, TABLE_OWNER, \\
							\> TABLE_NAME, TABLE_TYPE, UNIQUENESS, COMPRESSION)\\
				ALL_OBJECTS \> (OWNER, OBJECT_NAME, OBJECT_ID, DATA_OBJECT_ID, OBJECT_TYPE, CREATED, ...)\\
				ALL_SEQUENCES \> (SEQUENCE_OWNER, SEQUENCE_NAME, MIN_VALUE, MAX_VALUE, INCREMENT, CYCLE_FLAG)\\
				ALL_SYNONYMS \> (OWNER, SYNONYM_NAME, TABLE_OWNER, TABLE_NAME)\\
				ALL_TAB_COLUMNS \> (OWNER, TABLE_NAME, COLUMN_NAME, DATA_TYPE, DATA_LENGHT)\\
				ALL_TABLES \> (OWNER, TABLE_NAME, TABLESPACE_NAME)\\
				ALL_VIEWS \> (OWNER, VIEW_NAME, TEXT_LENGTH, TEXT,...)
			\end{tabbing}
		\end{alltt}
		
		Les tables de même nom préfixées par \texttt{USER\_} ont la même structure hormis l'attribut \texttt{OWNER}, et décrivent seulement les composants du schéma de l'utilisateur.
		
		\paragraph{Pseudo-colonnes}
			\begin{alltt}
				<nom_sequence>.\emph{CURRVAL}, <nom_sequence>.\emph{NEXTVAL}, \emph{LEVEL}, \emph{ROWID}, \emph{ROWNUM}, \emph{USER}
			\end{alltt}
			
\part{SQL comme langage de manipulation de données (LMD)}
	\section{Requêtes}
		\begin{alltt}
			\begin{tabbing}
				<requete> = \= \emph{SELECT} <liste_resultat | * >\\
							\> \emph{FROM} <liste_relations>\\
							\> [\emph{WHERE} <liste_conditions>]\\
							\> [\emph{GROUP BY} <liste_attributs_de_partitionnement>\\
							\> [\emph{HAVING} <liste_conditions_de_partitionnement>]]\\
							\> [\emph{ORDER BY} <liste_attributs_a_trier>]
			\end{tabbing}
		\end{alltt}
		où :
		\begin{alltt}
			\begin{tabbing}
				liste_resultat = \= [\emph{DISTINCT}] <attribut1 | expr1 | requete1> [<alias1>] \\
								 \> [,<attribut2 | expr2 | requete2> [<alias2>] ...\\
				liste_relations = \= <relation1 | vue1 | requete1> [alias1]\\
								  \> [, <relation2 | vue 2 | requete2> |alias2]...\\
				liste_conditions = [\emph{NOT}] <condition1> [\emph{AND} | \emph{OR} <condition2> ...]
			\end{tabbing}
		\end{alltt}
		
		Condition de sélection :
		\begin{alltt}
			<condition1> = <attribut [(+)] | expression> <comparateur | predicat_conditionnel> <constante>\\
			\begin{tabbing}
				<predicat_cond> = \= \emph{IS NULL} | \emph{IN} | \emph{BETWEEN} ... \emph{AND} | \emph{LIKE} \\
								  \> | \emph{IS NOT NULL} | \emph{NOT IN} | \emph{NOT BETWEEN} | \emph{NOT LIKE}
			\end{tabbing}
		\end{alltt}
		
		Condition de jointure prédicative :
		\begin{alltt}
			<conditionj> = <attribut1[(+)] | expr1> <comparateur> <attribut2[(+)] | expr2 >
		\end{alltt}
		
		Condition de jointure imbriquée :
		\begin{alltt}
			\begin{tabbing}
				<conditionji> = \= <expression1>[, <expression2>,...] \(\theta\) (<requete>)\\
								\> | <expression1>[, <expression2>,...] \(\theta\) \emph{ABY} | \emph{IN} (<requete>)\\
								\> | <expression1>[, <expression2>,...] \(\theta\) \emph{ANY} (<requete>)
			\end{tabbing}
		\end{alltt}
		
	\section{Calculs verticaux (fonctions aggrégatives)}
		\begin{alltt}
			<nom_fonction> ([\emph{DISTINCT}] <nom_colonne>)
		\end{alltt}
		où :
		\begin{alltt}
			<nom_fonction> = \emph{SUM} | \emph{AVG} | \emph{COUNT} | \emph{MAX} | \emph{MIN} | \emph{STDDEV} | \emph{VARIANCE}
		\end{alltt}
		
	\section{Tri des résultats}
		\begin{alltt}
			\emph{ORDER BY} <expression1> [\emph{ASC} | \emph{DESC}] [\emph{NULLS FIRST} | \emph{NULLS LAST}] [,<expression2> ...]
		\end{alltt}
		
	\section{Opérateurs ensemblistes}
		\begin{alltt}
			<requete1>
			\emph{UNION} | \emph{INTERSECT} | \emph{MINUS}
			<requete2>
		\end{alltt}
		
	\section{Test d'absence ou d'existence de données}
		\begin{alltt}
			\begin{tabbing}
				\emph{SELECT} \= <liste_attributs>\\
				\emph{FROM}   \> <relation1> [<alias1>] [,<relation2> [<alias2>] ...]\\
				\emph{WHERE} \> [<liste_conditions> \emph{AND} | \emph{OR}] [\emph{NOT}] \emph{EXISTS} (<sous_requete>)
			\end{tabbing}
		\end{alltt}
		
	\section{Classifications ou partitionnement}
		\begin{alltt}
			\begin{tabbing}
				\emph{ORDER BY}  \=<colonne1> [, <colonne2>,...]\\
				\emph{HAVING} 	\><liste_conditions_classe>
			\end{tabbing}
		\end{alltt}
		
	\section{Recherche dans une arborescence}
		\begin{alltt}
			\begin{tabbing}
				\emph{SELECT} \= <colonne1> [, <colonne2>, ...]\\
				\emph{FROM}   \> <table> [<alias>]\\
				[\emph{WHERE} \> <liste_conditions>]\\
				\emph{CONNECT BY} [\emph{PRIOR}] <colonne1> = [\emph{PRIOR}] <colonne2>\\
				[\emph{AND}	\> <condition_hierarchique>]\\
				[\emph{START WITH}\> <condition_depart>]\\
				[\emph{ORDER BY LEVEL}]
			\end{tabbing}
		\end{alltt}
		
	\section{Mise à jour des données}
		\begin{alltt}
			\begin{tabbing}
				\emph{UPDATE} \= <nom_table>\\
				\emph{SET} \> <nom_colonne1> = <expression1> [, <nom_colonne2> = <expression2>,...]\\
				[\emph{WHERE} \> <condition_selection>]
			\end{tabbing}
			\begin{tabbing}
				\emph{UPDATE} \= <nom_table>\\
				\emph{SET} \> (<nom_colonne1>[, <nom_colonne2>, ...]) = (\emph{SELECT} \= <colonne1>[,<colonne2>,...]\\
					\> \> \emph{FROM} ... \emph{WHERE} ...)\\
				[\emph{WHERE} \> <condition_selection>]
			\end{tabbing}
			\begin{tabbing}
				\emph{INSERT INTO} \= <nom_table> [(<liste_attributs>]\\
				\emph{VALUES} \> (<valeur1>[,<valeur2>,...])
			\end{tabbing}
			\emph{INSERT INTO} <nom_table> [(<liste_attributs>)] <requête>
			
			\emph{DELETE FROM} <nom_table> \emph{WHERE} <condition>
		\end{alltt}
\part{SQL comme langage de contrôle des données (LCD)}
	\section{Gestion des transactions}
		\begin{alltt}
			\emph{COMMIT, ROLLBACK}
		\end{alltt}
		
	\section{Création et suppression de roles et d'utilisateurs}
		\begin{alltt}
			\emph{CREATE ROLE} <nom_role> [\emph{IDENTIFIED BY} <mot_de_passe>]
			
			\emph{ALTER ROLE} <nom_role> [\emph{IDENTIFIED BY} <nouveau_mot_de_passe>]
			
			\emph{DROP ROLE} <nom_role>
			\begin{tabbing}
				\emph{CREATE USER} \= <nom_utilisateur> [\emph{IDENTIFIED BY} <mot_de_passe>]\\
				\emph{DEFAULT TABLESPACE} <nom_tablespace>\\
				\emph{QUOTA} \> <taille> \emph{PROFILE} <nom_profil>
			\end{tabbing}
			\emph{ALTER USER} <nom_utilisateur> [\emph{IDENTIFIED BY} <nouveau_mot_de_passe>]
			
			\emph{DROP USER} <nom_utilisateur>
		\end{alltt}
		
	\section{Attribution ou suppression de privilèges}
		\begin{alltt}
			\begin{tabbing}
				\emph{GRANT} \= <systeme_privileges | \emph{ALL} \emph{[PRIVILEGES}]>\\
				\emph{TO} \> <liste_role_utilisateur | \emph{PUBLIC}>\\
				[\emph{WITH ADMIN OPTION}]
			\end{tabbing}
		\end{alltt}
		
		où :
		\begin{alltt}
			\begin{tabbing}
				<systeme_privilege> = \= \emph{CREATE ROLE} | \emph{CREATE SEQUENCE} | \emph{CREATE SESSION} | \emph{CREATE SYNONYM}\\
				\>  | \emph{CREATE PUBLIC} | \emph{CREATE TABLE} | \emph{CREATE USER} | \emph{CREATE VIEW}
			\end{tabbing}
		\end{alltt}
		
		\begin{alltt}
			\begin{tabbing}
				\emph{GRANT} \= <liste_droits>\\
				\emph{TO} \> <nom_composant>\\
				[\emph{WITH GRANT OPTION}]
			\end{tabbing}
		\end{alltt}
		
		où:
		\begin{alltt}
			\begin{tabbing}
				<liste_droits> =\= \emph{SELECT} | \emph{INSERT} | \emph{UPDATE} [nom_colonne1,...] | \emph{DELETE} | \emph{ALTER} \\
							\> | \emph{INDEX} | \emph{REFERENCES} | \emph{ALL} [\emph{PRIVILIGES}]
			\end{tabbing}
		\end{alltt}
		
		\begin{alltt}
			\begin{tabbing}
				\emph{GRANT} \= <liste_roles_attribues>\\
				\emph{TO} \> <liste_roles_utilisateurs>\\
				[\emph{WITH ADMIN OPTION}]
			\end{tabbing}
		\end{alltt}
		
	\section{Gestion de vues}
		\begin{alltt}
			\emph{CREATE} [\emph{OR REPLACE}] [[\emph{NO}] \emph{FORCE}]
			\emph{VIEW} <nom_vue> [(alias1[,alias2,...])]
			\emph{AS} <requete>
			[\emph{WITH CHECK OPTION} | \emph{WITH READ ONLY}]
			
			\emph{ALTER VIEW} <nom_vue> \emph{COMPILE}
			
			\emph{DROP VIEW} <nom_vue>
		\end{alltt}
\end{document}